\documentclass{article}
\input{preamble.tex}

\title{Assignment 3 DISYS}
\author{frgm, mhsi & sibh}
\date{\today}


\begin{document}
\maketitle

\section*{System architecture}
Our system uses a server-client architecture.
This means that each client only talks to the server,
and the server is therefore responsible for handling joining and leaving clients.
The clients therefore also don't need to know anything about the other client.
When a client sends a message, it is sent to the server which then distributes the message to all clients.

\section*{RPC Methods}
We have one method in the .proto 

\section*{Lamport timestamps}
We have implemented Lamport timestamps by storing a counter at each client, which the increases by one before sending a message, and after recieving a message. This is explains why the timestamp increases by two each time a message is sent.
\\
\\ 
When recieving a message, we take the maximum value of the local timestamp and the timestamp of the incoming message, increases it by one, and keeps it as new the local timestamp.

\section*{Traces of RPC calls}

\end{document}